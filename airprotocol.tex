\chapter{Air Protocols}
\section{Introduction}

The data collection end of the OpenAMD project
uses {\em tags} and {\em readers}.
Tags serve to instrument objects
-- attendees --
in the real world and report things about them.
Readers listen for these reports
and forward them to an {\em aggregator},
see \autoref{sec:aggregator}.  In the future,
readers may be more powerful,
doing analysis or multi-user pub-sub;
for the moment,
they simply pass their data
to the singleton aggregator.
Tags are very simple devices and currently
only beacon with their identifiers and
report the status of a few inputs.
Transmit-only tag behavior
helps to minimize power draw
and increase the useful lifetime of tag batteries.

The OpenAMD air protocols have a short stack of containers:
there's an outer, packetizing link layer transport
({\it e.g.}, nRF24 protocol, UDP/IP)
and an inner TLV (Type-Length-Value) stream.
We require that TLV entries fit inside packet boundaries.

\section{Link Layer Transports}

\subsection{nRF}

The nRF link layer is a little funny.
The documentation does not describe
how to put chips into promiscuous mode,
and the host-facing SPI transport does not include
the destination MAC address of packets it receives,
only the ``MultiCeiver\texttrademark~ channel''.
MultiCeiver\texttrademark~ channels are configurable oddly:
channel 0 has its own address,
and channels 1 through 5 have their prefix bytes in common.
There is no such thing as a source address on the air.
We imagine that the intended use of this addressing scheme was to
allow for a singular unicast address and many multicasts all using
a common prefix, but it still strikes us as a little odd.

In addition to knowing which MAC a given network is using,
one has to know its channel (frequency)
and the flavor of checksumming used on the radio to filter out noise, if any.

OpenAMD uses the MAC to distinguish protocols (TLV streams) being encapsulated.
Currently, OpenAMD defines the channel/MAC pairs shown in \autoref{fig:airpro:nrfmac}.

\begin{figure}[p]
    \begin{center}\begin{tabular}{lllll}
        Channel & MAC                       & CRC     & Protocol & XREF \\
        81      & 0x424541434F (``BEACO'')  & 1 byte
            & OpenBeacon Beacon Container & \autoref{sec:airpro:openbeacon} \\
        81      & 0x0102030201              & 1 byte
            & OpenBeacon OpenBouncer      & \autoref{sec:airpro:openbouncer} \\
    \end{tabular}\end{center}
    \caption{Reserved OpenAMD nRF encapsulation schemes}
    \label{fig:airpro:nrfmac}
\end{figure}

\subsection{OpenBeacon OpenBouncer Protocol}
\label{sec:airpro:openbouncer}

The OpenBeacon subproject OpenBouncer uses
a different format on the air altogether
and is not documented here
save to reserve its nRF MAC address.

\subsection{UDP/IP}

This is pretty bog standard.
We strip off the nRF container
and plop the data into a UDP packet.
The source address of the packet
is the node doing the decapsulation.
For OpenBeacon TLV streams,
we adopt use of UDP port 2342.

\section{OpenBeacon TLV}

This framing scheme is stolen from OpenBeacon \cite{openbeacon}.
We believe this is the first attempt to describe it formally
and centrally collect the types used internally.

OpenBeacon TLV packets
are framed by a two byte header,
consisting of the length and type byes,
and are terminated with a CRC-16 checksum.
The length byte includes itself,
the type byte,
and the two-byte checksum.
See \autoref{fig:airpro:obframe}.
All multi-byte fields 
are in Network Endian order
({\it i.e.} MSB first).
Type values are assigned
as in \autoref{fig:airpro:obtypes}.
The OpenAMD project
will attempt to coordinate
any additional assignments
with the OpenBeacon project.
Reserved OpenBeacon packet types
have been marked as such,
in an effort to avoid conflicts.

\begin{figure}[p]
    \begin{center}\begin{bytefield}{16}
        \bitheader{0,7-8,15}\\
        \bitbox{8}{Length} & \bitbox{8}{Type} \\
        \skippedwords \\
        \bitbox{16}{CRC-16}
    \end{bytefield}\end{center}
    \caption{OpenBeacon TLV Framing}
    \label{fig:airpro:obframe}
\end{figure}

\begin{figure}[p]
    \begin{center}\begin{tabular}{ll}
        Type & Payload \\
		{\tt 0x16} & Reserved (unused; old OpenBeacon protocol) \\
        {\tt 0x17} & OpenBeacon Tracker Report \\
		{\tt 0x18} & (Obsolete) OpenBeacon Tracker Report \& older OpenBeacon protocols \\
		{\tt 0x19} & Reserved (unused; old OpenBeacon protocol) \\
		{\tt 0x2A} & Reserved (unused; old OpenBeacon protocols) \\
		{\tt 0x45} & Reserved (unused; old OpenBeacon protocol) \\
    \end{tabular}\end{center}
    \caption{Assigned OpenBeacon TLV Types.}
    \label{fig:airpro:obtypes}
\end{figure}

\subsection{Backwards Compatibility with OpenBeacon Data}

Unfortunately OpenBeacon was not consistent
in their assignments of type fields.
At various points in the past,
the following types have been in use
for one or more protocols.
We list parenthetically the mnemonics
used in various parts of the source tree
for each type identifier.
\begin{itemize}
    \item {\tt 0x16} ({\tt RFBPROTO\_READER\_ANNOUNCE})
    \item {\tt 0x17} ({\tt RFBPROTO\_READER\_COMMAND}, {\tt RFBPROTO\_BEACONTRACKER})
    \item {\tt 0x18} ({\tt RFBPROTO\_BEACONTRACKER}, {\tt RFBPROTO\_BEACONNODE})
    \item {\tt 0x19} ({\tt RFBPROTO\_BEACONVOTESTATS})
    \item {\tt 0x2A} ({\tt RFBPROTO\_PROXTRACKER}, {\tt RFBPROTO\_BLINKENLIGHTS})
    \item {\tt 0x45} ({\tt RFBPROTO\_PROXREPORT})
\end{itemize}
Further, OpenBeacon was not consistent in layout of data
even within type mnemonic.
For example, there are two different
structures implied by {\tt RFBPROTO\_BEACONTRACKER}
(one for value {\tt 0x17}, and one for {\tt 0x18}).

This unfortunate result probably came about by lack of
a central registry of identifiers and structures.
For those eager to decode old badge beacons,
we {\em believe} that
packet types other than {\tt RFBPROTO\_BEACONTRACKER}
are rare in the wild.
See \autoref{sec:airpro:openbeacontrackback}

\section{OpenBeacon Protocols}

\subsection{{\tt 0x17} : OpenBeacon Tracker Report}
\label{sec:airpro:openbeacon}

This protocol is stolen
from the OpenBeacon project \cite{openbeacon}.

The data inside a Tracker Report
is shown in \autoref{fig:airpro:obtrp}.
(The length field
for this kind of packet
is always {\tt 0x10}.)

The {\tt SENSE} field is the
result of reading on-board sensors,
and varies
from instantiation to instantiation.
Parsers are encouraged
to look more for transitions
than at the actual values;
attempts to divine
any particular meaning
from {\tt SENSE} values
may be ill-advised
without some certainty
about the originating hardware.

The transmission strength ({\tt TX STR}) data is
an indicator of how loudly the packet was sent;
since the nRF chips do not
give an indication of RSSI,
the beacons are expected
to vary their transmission power with time
to provide the receivers
with some approximate idea
of link quality.


\begin{figure}[p]
    \begin{center}\begin{bytefield}{16}
        \bitheader{0,7-8,15}\\
        \bitbox{8}{\tt SENSE} & \bitbox{8}{\tt TX STR} \\
        \bitbox{16}{\tt Seq Hi} \\
        \bitbox{16}{\tt Seq Lo} \\
        \bitbox{16}{\tt Device ID Hi} \\
        \bitbox{16}{\tt Device ID Lo} \\
        \bitbox{16}{{\tt RESERVED (0x00)}} \\
    \end{bytefield}\end{center}
    \caption{OpenBeacon Tracker Report Payload}
    \label{fig:airpro:obtrp}
\end{figure}

\subsubsection{Backwards Compatibility with OpenBeacon Tracker Reports}
\label{sec:airpro:openbeacontrackback}

OpenBeacon chooses to encrypt the entire network
with XXTEA \cite{needham:xtea,wheeler:xxtea}
using a single, static, published key per event.
Unfortunately, the encryption covers
the TLV headers
as well as the payload.
Therefore,
as an extremely special case,
in the spirit of hardware reuse,
we recommend
attempting to decode,
using one or more of these keys
(republished in \autoref{fig:airpro:xxteakeys}
as an act of neighborly good will),
OpenBeacon TLV streams which fail to frame,
and, if the result frames properly,
parse that instead.

The data layout chosen above corresponds with
OpenBeacon's data layout exactly when
{\tt RFBPROTO\_BEACONTRACKER} had value {\tt 0x17}.
The other packet format, corresponding to {\tt 0x18},
is described in \autoref{fig:airpro:oboldtracker}
solely for the purpose of backwards compatibility.
Attempts to frame packets of this shape must only
be made when attempting to fall back and XXTEA
decryption has been applied.

Further, note that the field labeled {\tt SENSE}
in \autoref{fig:airpro:obtrp} was, in OpenBeacon firmwares,
a collection of flags
which varied by firmware.
% TODO

\begin{figure}[p]
    \begin{center} \begin{tabular}{cl}
        Key & Used At \\
        {\tt 0x7013F569,0x4417CA7E,0x07AAA968,0x822D7554} & 25C3 free beta version key \\
        {\tt 0x8e7d6649,0x7e82fa5b,0xddd4541e,0xe23742cb} & Camp 2007 key \\
        {\tt 0x9c43725e,0xad8ec2ab,0x6ebad8db,0xf29c3638} & The Last Hope - AMD \\
        {\tt 0xab94ec75,0x160869c5,0xfbf908da,0x60bedc73} & ? \\
        {\tt 0xb4595344,0xd3e119b6,0xa814d0ec,0xeff5a24e} & 24C3 key \\
        {\tt 0xbf0c3a08,0x1d4228fc,0x4244b2b0,0x0b4492e9} & 25C3 final key \\
        {\tt 0xdeadbeef,0xdeadbeef,0xdeadbeef,0xdeadbeef} & 23C3 ? \\
        {\tt 0xe107341e,0xab99c57e,0x48e17803,0x52fb4d16} & 23C3 key \\
        {\tt 0xee2522d1,0xdbc221f1,0xa21d0d0e,0x865976a2} & ? \\
    \end{tabular} \end{center}
    \caption{XXTEA keys used by OpenBeacon installations at times past.}
    \label{fig:airpro:xxteakeys}
\end{figure}

\begin{figure}[p]
    \begin{center}\begin{bytefield}{16}
        \bitheader{0,7-8,15}\\
        \bitbox{8}{\tt TX STR} & \bitbox{8}{\tt Device ID Hi} \\
        \bitbox{8}{\tt Device ID Lo} & \bitbox{8}{\tt Boot Count Hi} \\
        \bitbox{8}{\tt Boot Count Lo} & \bitbox{8}{\tt RESERVED} \\
        \bitbox{16}{\tt Sequence Count Hi} \\
        \bitbox{16}{\tt Sequence Count Lo} \\
    \end{bytefield}\end{center}
    \caption{Obsolete OpenBeacon Tracker Report Payload}
    \label{fig:airpro:oboldtracker}
\end{figure}


\section{Extending The Air Protocols}

We encourage you to experiment!
Please run your experiments on
nRF MAC addresses
other than those defined in \autoref{fig:airpro:nrfmac}.
If you aim to design a protocol that
will be merged back into the OpenAMD suite,
please honor the TLV container format.
On unofficial MACs,
type values are unconstrained;
when protocols merge,
they will be assigned
nonconflicting type values
and this document will be updated.

We encourage you to preserve
OpenAMD beacon functionality of tags
when reprogramming them with alternate firmwares.
Since the nRF chips are able
to listen and transmit on multiple addresses,
and OpenAMD firmwares do not currently
use the common-prefix-channels,
we suggest leaving channel $0$
set to the OpenAMD MAC.
